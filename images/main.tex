% LaTeX resume using res.cls
\documentclass[margin]{res}
\usepackage{helvet} % uses helvetica postscript font (download helvetica.sty)
%\usepackage{newcent}   % uses new century schoolbook postscript font 
\setlength{\textwidth}{6.2in} % set width of text portion
\newcommand{\code}[1]{\texttt{#1}}
\usepackage{hyperref}
\hypersetup{
    colorlinks=true,       % false: boxed links; true: colored links
    linkcolor=red,          % color of internal links (change box color with linkbordercolor)
    citecolor=green,        % color of links to bibliography
    filecolor=magenta,      % color of file links
    urlcolor=blue           % color of external links
}

\begin{document}

% Center the name over the entire width of resume:
 \moveleft.5\hoffset\centerline{\textsc{\Large Krzysztof Suberlak}}
 %\moveleft.5\hoffset\centerline{Curriculum Vitae}

% Draw a horizontal line the whole width of resume:
%\moveleft\hoffset\vbox{\hrule width\resumewidth height 1pt}\smallskip
% \moveleft.5\hoffset\centerline{Email: \href{mailto:suberlak@uw.edu}{suberlak@uw.edu}}
% \moveleft.5\hoffset\centerline{Website: \url{http://suberlak.github.io/}}
\moveleft\hoffset\vbox{\hrule width\resumewidth height 1pt}

\hspace{-1in}
\begin{minipage}[t]{0.95\resumewidth}
Email: \href{mailto:suberlak@uw.edu}{suberlak@uw.edu} \hfill GitHub: \href{https://github.com/suberlak}{suberlak}\\
Web: \url{suberlak.github.io/} \hfill 
Mobile: 206-915-9093
\end{minipage}
%% Select character to use as the bullet marker. 
%% Comment out to get normal dark filled circle bullets
\def\labelitemi{--}	

\begin{resume}

\section{Interests} Astrostatistics and data mining. I study how quasar properties such as black hole mass, or accretion disk luminosity, correlate with damped random walk parametrization of their light curves.
%I study how quasars can be classified and characterized based on their variability. %I use Gaussian Processes to fit quasars light curves using the damped random walk description, and correlate 

%I investigate the correlations between damped random walk  of the light curves and the physical properties of quasars, such as black hole mass or quasar luminosity.  
%
%I study how physical properties of quasars, such as black hole mass or accretion disk luminosity, correlate with the Damped Random Walk parametrization of observed light curves. 




\section{Education} 
University of Washington, Seattle, WA \hfill 2013 -- present\\
PhD Candidate in Astronomy (expected graduation 2019)

University of Oxford, UK \hfill 2008 -- 2012\\
MPhys Physics

\section{Computer skills}
\textbf{Python} open data science stack (NumPy, SciPy, AstroPy, Pandas, Matplotlib, Scikit-learn, iPython, Jupyter-Lab, AstroML, etc.); 
Github (\textbf{version control});
\textbf{UNIX} based systems;
\textbf{LSST science pipelines}; 
Database manipulation: \textbf{SQL}, Apache-Spark, AXS, Dask, LSD; 
Collaboration tools: \textbf{Jira}, \textbf{Confluence}, Docushare, \textbf{LaTeX}, Zenodo.


\section{Selected \\Graduate \\Research} 
{\sl LSST Crowded Fields}:  DM Subsystem Science Team  \hfill          2018\\
Comparing the results of LSST stack processing of DECAPS data to the state-of-the-art pipeline in areas of high stellar density 
\begin{itemize}  \itemsep -2pt %reduce space between items
\item Analyzed the processed images and source catalogs, identified figures of merit
\item Made recommendations concerning photometric accuracy and astrometric precision (DMTN077 ``\href{https://dmtn-077.lsst.io}{LSST Fall 2017 Crowded Fields Testing}'')
\end{itemize}  

{\sl LSST Prototype Data Access Center}: DM Subsystem Science Team \hfill            2017
\begin{itemize}  \itemsep -2pt 
\item Tested the functionality of PDAC
\item Made recommendations for the DM-SST team, summarized in the report DMTR022 ``\href{https://ls.st/DMTR-22}{Prototype Data Access Center: User Report}'' 
\end{itemize}

{\sl eScience Data Science for Social Good} \hfill Jun 2015 -- Aug 2015\\
 Summer work at the University of Washington eScience Institute, with Dr. Ariel Rokem and Dr. Bryna Hazelton  on a Gates Foundation project ``\href{https://escience.washington.edu/get-involved/data-science-for-social-good/dssg-project-summaries-15}{Predictors of Permanent Housing for Homeless Families}''
\begin{itemize}  \itemsep -1pt 
\item Cleaned the heterogeneous datasets describing homeless shelters in King, Pierce and Snohomish counties
\item Developed python code with \href{https://uwescience.github.io/DSSG2015-predicting-permanent-housing/2015-07-27-chris-galaxy-clusters/}{hierarchical clustering}  to define families based on coincidence of entry times and IDs
\end{itemize}

%\section{Employment} 
%{\sl University of Washington, Research Assistantship} \hfill           Jan 2016 -- present\\ Graduate Research Assistantship with Dr. \v{Z}eljko Ivezi\'c \
%{\sl University of Washington, Teaching Assistantship} \hfill           Oct 2013 -- Dec 2015 \\ Graduate Teaching Assistantship

\section{Under-\\graduate\\Research} 

{\sl Nicolaus Copernicus Astronomical Center, Poland, Research Associate} \hfill Feb 2013 -- Jul 2013 \\ Research at the Polish Academy of Sciences  with Dr. Agata R\'{o}\.{z}a\'{n}ska
\begin{itemize}
	\item Measured Active Galactic Nuclei spectra from the VIMOS Public Extragalactic Redshift Survey
	\item Improved classification scheme and data reduction software
\end{itemize}

{\sl University of Oxford, Research Studentship} \hfill Oct 2012 -- Dec 2012 \\ Research with Dr. Leigh Fletcher and Prof. Pat Irwin
\begin{itemize}
	\item Analyzed the infrared data of Jupyter atmosphere from Cassini 
	\item Verified the possible  depth of measurement using ethane spectral lines
\end{itemize}


{\sl Nicolaus Copernicus Astronomical Center, Poland, Summer Internship}  \hfill   Jun 2012 -- Aug 2012 \\
Research at the Polish Academy of Sciences with Dr. Agata R\'{o}\.{z}a\'{n}ska
\begin{itemize}
	\item Analyzed Chandra x-ray data, performed  spectroscopy and imaging of Sagittarius A*
	\item Investigated the spectroscopy of x-ray filaments, and examined the morphology of the region in various energy bands
\end{itemize}


{\sl University of Oxford,  Masters Thesis} \hfill            Jan 2012 -- Apr 2012 \\
Measuring Expansion of the Universe with Supernovae with Dr. Fraser Clarke and Dr. Mark Sullivan
\begin{itemize}
\item Observed, reduced, and analysed follow-up data on newly discovered supernovae using the  Oxford Wetton telescope
\item Measured the Hubble constant with the lightcurve fitting software 
\end{itemize} 



{\sl University College of London, Nuffield Fellowship} \hfill   Jun 2011 -- Aug 2011 \\
Undergraduate Research at the Mullard Space Science Laboratory, UK, with  Prof Andrew Coates and  Dr. Adam Masters
\begin{itemize}%  \itemsep -2pt %reduce space between items
\item Analyzed the location of Saturn's ‘plasmapause’ using Cassini Plasma Spectrometer (CAPS) Electron Spectrometer (ELS) data” 
\end{itemize}


{\sl University of Oxford, AOPP Research Assistantship} \hfill            Jun 2010 -- Aug 2010 \\
Summer research internship with Dr. Neil Bowles and Dr. Ian Thomas  at the University of Oxford Oceanic and Planetary Physics sub-department
\begin{itemize}%  \itemsep -2pt %reduce space between items
\item Performed laboratory measurements and data analysis supporting the Diviner instrument on the Lunar Reconnaissance Orbiter
\item Determined the grain size distribution of the lunar soil equivalent, to aid modelling of thermal emission of lunar regolith
\end{itemize}  




\section{Publications}
\begin{itemize}   
\item \textbf{Suberlak, K.L.}, Ivezi\'c, \v{Z}., MacLeod, C.L., Graham, M., Branimir, S. ``\href{https://doi.org/10.1093/mnras/stx2310}{Solving the puzzle of discrepant quasar variability on monthly time-scales implied by SDSS and CRTS data sets}.'' Monthly Notices of the Royal Astronomical Society, Volume 472, Issue 4, p.4870-4877 (2017)

\end{itemize}


\section{Honors\\And\\Awards} 
\begin{itemize}  

\item Data Intensive Research in Astrophysics and Cosmology (DIRAC) at the University of Washington: \href{https://dirac.astro.washington.edu}{DIRAC Institute Fellow }(2016-present) 
\item University of Washington eScience Institute \href{https://escience.washington.edu/dssg/}{Data Science for Social Good Fellow} (2015-present)
\item Fellow of the Royal Astronomical Society (2008-present) \\
\end{itemize}
                 


\section{Professional\\ Presentations} 
\begin{itemize}  %\itemsep -5pt %reduce space between items

\item Poster:    Astrophysical Frontiers in the Next Decade and Beyond: Planets, Galaxies, Black Holes,~\& the Transient Universe. Portland, OR. June 26, 2018 

\item Poster: ``Bayesian inference in forced photometry'' at \href{http://myweb.facstaff.wwu.edu/~davenpj3/nwam2016/}{Northwest Astronomy Meeting}, Bellinham, WA. Oct 29, 2016 

\item Poster: ``What to do with negative fluxes?'' at the intermediate Palomar Transient Factory (iPTF) Summer School, California Institute of Technology.  Pasadena, CA. July 18, 2016

\item Poster: ``\href{http://adsabs.harvard.edu/abs/2016AAS...22724335S}{Solving the puzzle of discrepant quasar variability on monthly time-scales implied by SDSS and CRTS datasets}.'' 227\textsuperscript{th} American Astronomical Society Meeting. Kissimmee, FL. January 6, 2016.

\item Poster: ``\href{http://adsabs.harvard.edu/abs/2015AAS...22514464S}{New Constraints on Quasar Variability based on 8,000 SDSS Stripe 82 Quasars with both SDSS and CRTS Lightcurve Data}.'' 225\textsuperscript{th} American Astronomical Society Meeting. Seattle, WA. January 6, 2015.

\end{itemize}


\section{Workshops and Conferences}
\begin{itemize}

    \item LSST 2017 Project and Community Workshop. Tucson, AZ. Aug 14-18, 2017

	\item \href{http://www.cvent.com/events/detecting-the-unexpected-discovery-in-the-era-of-astronomically-big-data/event-summary-0db6808d548b4a9ea6466b43046a1ff5.aspx}{Detecting the Unexpected: Discovery in the Era of Astronomically Big Data}. Space Telescope Science Institute, Baltimore, MD. Feb 27 - March 2, 2017 

	\item \href{https://www.imprs-hd.mpg.de/147422/Summer-School-2016}{Summer School 2016 Astrostatistics \& Data Mining}. International Max Planck Research School for Astronomy \& Cosmic Physics at the University of Heidelberg, Germany. Sept 12-16, 2016

\end{itemize}


\section{Teaching \\Experience}
\begin{itemize}   
	
\item ASTR150 The Planets: Teaching assistant for  three quarters (Winter 2013, Summer 2014 for Dr Nicole Silvestri; Spring 2015 for Dr Toby Smith)

\item ASTR101 Introduction to Astronomy: Teaching assistant for  eight quarters (Fall 2013, Fall 2015, Summer 2016, Autumn 2016 for Dr Ana Larson ; Spring 2014, Spring 2016 for Dr Chris Laws; Winter 2015, Winter 2016 for Dr Oliver Fraser) \\
\end{itemize}

%\section{References}
%\begin{itemize}
%\item Dr.~Mario Juri\'c 609-933-1033 \href{mailto:mjuric@astro.washington.edu}{mjuric@astro.washington.edu}
%\item Prof.~Andy Connolly 206-543-9541 \href{mailto:ajc@astro.washington.edu}{ajc@astro.washington.edu}
%\item Dr.~Colin Slater 206-685-9017 \href{mailto:ctslater@uw.edu}{ctslater@uw.edu}  
%\end{itemize}

\vfill \hfill {\small Last updated: \today}
\end{resume}
\end{document}